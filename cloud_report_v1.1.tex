\documentclass[12pt,a4paper,oneside]{article}
\usepackage{hyperref,tfrupee,amsmath}
\usepackage[tablegrid]{vhistory}

\hypersetup{
	pdftitle={Comparison between cloud and traditional deployment for MIS,
		NIT Calicut},
	pdfsubject={Cloud and traditional deployment for MIS, NIT Calicut},
	pdfauthor={Atul Golchha, Arpan Kapoor, Bobba Praveen, G. V. S. Rahul,
			Sachin T. Sany},
	hidelinks
}

\begin{document}
\title{Comparison between cloud and traditional deployment for MIS, NIT Calicut}
\author{Atul Golchha (AG)
	\\
	Arpan Kapoor (AK)
	\\
	Bobba Praveen (BP)
	\\
	G. V. S. Rahul (GVSR)
	\\
	Sachin T. Sany (STS)
	\\ \\
	BTech, Computer Science \& Engineering (2012--2016)}


\maketitle

% Change width ratios for Authors & Description columns in the version history table.
% When changing the widths, please take care that the widths add up to 2\hsize.
\renewcommand \vhAuthorColWidth{.8\hsize}
\renewcommand \vhChangeColWidth{1.2\hsize}

\begin{versionhistory}
	\vhEntry{1.0}{22.04.15}{AG|AK|BP|GVSR|STS}{Initial version}
	\vhEntry{1.1}{24.04.15}{Dr. Vinod Pathari}{Updated \emph{Conclusion}}
\end{versionhistory}

% No need for page number on the first page
\pagenumbering{gobble}

\newpage
% Restart page numbering
\pagenumbering{arabic}
\tableofcontents
\newpage

\section{Introduction}
National Institute of Technology, Calicut (NIT Calicut) is one of the leading
research and academic institutes in engineering and science disciplines in our
country. This document compares between a traditional and a cloud deployment
for a Management Information System (MIS) at NIT Calicut. It is prepared as an
independent study, with the support of Dr. Vineeth Paleri and Dr. Vinod Pathari,
reflecting on the booming transition from the traditional software deployment
model to the cloud model.

Before proceeding with the comparison, we will have a look at the current
system followed. Then we will describe the characteristics of a traditional and
cloud deployment. Finally we will compare the two systems for MIS.

\subsection{Current System}
Currently, we have a management software called Decision Support System (DSS).
Below are some of the details about DSS, as of April, 2015:

\begin{itemize}
	\item Purchased in 2004 from \emph{Master Software, Nagpur} for around
		\rupee 16 lac.
	\item Used by all students, academic and non-academic staff.
	\item Supposed to handle usage by around 1500 students, the college
		strength in 2004. Current strength is 6000.
	\item Had a warranty period of 3 years. After the warranty, we
		maintained an AMC with the company till 2011. The contract was
		canceled in 2011 due to high AMC demanded by the company.
	\item Since 2011, 1 permanent and 2 ad hoc staff have been appointed for
		its maintenance.
	\item Daily backup has to be performed by the staff.
	\item Current server specification:
	\begin{itemize}
		\item 2 servers - 1 database server (Oracle 9i) and 1 web
			server.
		\item OS - Windows 2003 Server (32-bit).
		\item RAM - 16GB (only 4GB usable, as per the staff).
	\end{itemize}
	\item Upgrading to Oracle 10g (database) is technically challenging, as
		per the staff.
	\item 2 new servers (32GB RAM, 2TB HDD) purchased in 2015 for upgrading
		the old DSS servers.
\end{itemize}

\section{Traditional solution}
\begin{itemize}
	\item With traditional hosting, the software is deployed on dedicated
		servers owned by the client.
	\item As time passes, hardware performance degrades and thus requires
		upgrades.
	\item The client pays the software provider for its maintenance.
	\item Generally, a single unmanaged backup scheme is used.
	\item Client has to take the burden of hardware maintenance like
		machines, UPS, network equipments, etc.
	\item There is limited remote access possibilities with added cost and
		security issues.
\end{itemize}

\section{Cloud solution}
According to the NIST \cite{ref0}, Cloud computing is:
\begin{quote}
\emph{a model for enabling ubiquitous, convenient, on-demand network access to
a shared pool of configurable computing resources (e.g., networks, servers,
storage, applications, and services) that can be rapidly provisioned and
released with minimal management effort or service provider interaction.}
\end{quote}

\noindent
It identifies five \textbf{essential characteristics} of the cloud model:
\begin{enumerate}
	\item On-demand self-service
	\item Broad network access
	\item Resource pooling
	\item Rapid elasticity (scalability)
	\item Measured service
\end{enumerate}

\subsection{Classification}
There are 3 basic cloud computing service models. A very simplified way of
differentiating these flavors of Cloud Computing is as follows:
\begin{itemize}
	\item SaaS applications are designed for end-users, delivered over the
		web.
	\item PaaS is the set of tools and services designed to make coding and
		deploying those applications quick and efficient.
	\item IaaS is the hardware and software that powers it all -- servers,
		storage, networks, operating systems.
\end{itemize}

\subsubsection{Software as a Service}
With SaaS, a provider licenses an application to customers as a service on
demand, through a subscription. Some of the defining characteristics of SaaS
include:
\begin{itemize}
	\item Web access to software.
	\item Software is managed from a central location by the provider.
	\item Users not required to handle software upgrades and patches.
	\item The client does not own the hardware and hence no maintenance
		issues arise.
\end{itemize}
Among the most familiar SaaS applications for business are customer
relationship management applications like Salesforce, productivity software
suites like Google Apps and Microsoft Office 365, and storage solutions like
Box and Dropbox.

\subsubsection{Platform as a Service}
In the PaaS model, cloud providers deliver a computing platform, typically
including operating system, programming language execution environment,
database, and web server. Application developers can develop and run their
software solutions on a cloud platform without the cost and complexity of
buying and managing the underlying hardware and software.

Examples of PaaS providers include Heroku, Google App Engine, Microsoft Azure
and Red Hat's OpenShift.

\subsubsection{Infrastructure as a Service}
IaaS is a way of delivering Cloud Computing infrastructure –- servers, storage,
network, and operating systems – as an on-demand service. Rather than purchasing
servers, software, storage space or network equipments as in a traditional
system, clients instead buy those resources as a fully outsourced service on
demand.
\begin{itemize}
	\item Resources are provided as a service.
	\item Allows for dynamic scaling.
	\item Has a variable cost, utility pricing model.
	\item Generally includes multiple users on a single piece of hardware.
\end{itemize}
Some of the popular IaaS providers include Amazon Web Services, Google Compute
Engine, Rackspace Open Cloud and IBM SmartCloud Enterprise.

\subsection{Data Migration}
\subsubsection{Entry}
Cloud providers generally accommodate multiple formats to import client data.
\subsubsection{Exit}
To prevent vendor lock down, the client should be able to export his data in a
non proprietary format like CSV, XML, etc. These constraints should be
specified in the software requirement specification by the client.

\subsection{Reliability}
The reliability depends upon the cloud provider. At data centers, generally
professionals are employed who monitor 24\(\times\)7 and thus guarantee higher
availability and minimum impact of any failures.

\subsection{Security}
When an organization elects to store data or host applications on the cloud,
it loses its ability to have physical access to the servers hosting its
information. As a result, potentially business sensitive and confidential data
is at risk from insider attacks.

In order to conserve resources, cut costs, and maintain efficiency, Cloud
Service Providers often store more than one customer's data on the same server.
Hence, the data isolation aspect should be kept in mind when choosing the cloud
provider.

\subsection{Cost}
Generally, the cloud model has an annually recurring expenditure, which has to
be clearly stated at the point of choosing the cloud provider.

The current MIS SRS states the requirements at a very high level, thus a
precise estimate of the cost cannot be made.

\subsection{Adoption in India}
There are over 8000 companies in India, as of 2015, that are using cloud
solutions, including large enterprises such as Tata Motors, Reliance
Entertainment, NDTV, Narayana Health, Macmillan India, EROS International,
Malayala Manorama and Sony Entertainment \cite{ref6}.

The Government of India's UID (Aadhar) project is entirely based on cloud
technology \cite{ref7}.

\section{Comparison}
\subsection{Ownership}
In a traditional system, the client owns the software and data resides on the
client's server. When it comes to cloud, there are 3 service models viz. SaaS,
PaaS and IaaS. In SaaS, client doesn't have the ownership, whereas in PaaS and
IaaS, there is possibility for the client to have ownership of the software.

In case of client ownership, maintenance can either be outsourced to a company
or client can hire their own professionals.

In our case, it would be difficult to hire technically knowledgeable
professionals, so possessing ownership doesn't have any significant advantages.
In this regard, SaaS is preferred.

\newpage
\subsection{Comparison Table}
The following table assumes the SaaS service model for the cloud solution. \\
\\
% Increase space between rows
\renewcommand{\arraystretch}{1.5}
\begin{tabular}{|p{2.26cm}|p{5cm}|p{5.4cm}|}
	\hline
	\textbf{Topic} & \textbf{Traditional} & \textbf{Cloud (SaaS)}
	\\ \hline
	Ownership &
	Client owns &
	No ownership
	\\ \hline
	Data \newline migration &
	Entry and exit follows \newline standard formats &
	Possible charge for data \newline export
	\\ \hline
	Reliability &
	24\(\times\)7 monitoring not \newline possible, higher Mean Time To
		Recovery (MTTR) &
	24\(\times\)7 monitoring done, lower MTTR
	\\ \hline
	Security &
	Client's responsibility for physical security &
	Chances of \emph{insider's attack} and \emph{improper data isolation}
	\\ \hline
	Manpower &
	Trained professionals \newline required to run and \newline maintain the
		system &
	Requires minimum manpower for administrative purposes
	\\ \hline
	Scalability &
	Not rapidly scalable &
	Can easily meet higher service requirements at peak times
	\\ \hline
	Hardware maintenance &
	Burden of hardware \newline maintenance &
	No maintenance
	\\ \hline
	Software maintenance &
	Supporting software may require upgrades &
	Upgrades are transparent to the client
	\\ \hline
	Cost &
	\multicolumn{2}{c|}{Unable to estimate due to lack of complete SRS}
	\\ \hline
\end{tabular}

\newpage
\section{Conclusion}
Choosing between cloud and traditional deployment is hard without the complete
specification for MIS. Ownership of the software is beneficial only if
professionals can be hired for its maintenance, which is difficult in our case.
Hence it is recommended that a cloud solution would be better for NITC MIS.

In the cloud deployment options, Software as a Service (SaaS) is the suggested
choice. The following two points must be studied and well-addressed while opting
a SaaS model:
\begin{itemize}
	\item Data format for possible migration should be specified upfront.
		The suggested format should be in-line with the industry
		standards to ensure easy migration as and when NITC chooses to
		exit from the service provider.
	\item The service level agreement (SLA) should clearly specify
		quantifiable parameters for reliability, scalability,
		maintenance etc. with associated financial implications, where
		necessary.
\end{itemize}

\newpage
\begin{thebibliography}{99}
\bibitem{ref0}
	National Institute of Standards and Technology,
	\emph{The NIST Definition of Cloud Computing,}
	\url{http://csrc.nist.gov/publications/nistpubs/800-145/SP800-145.pdf}
\bibitem{ref1}
	Agility Networks,
	\emph{Cloud Computing vs Traditional Server,}
	\url{http://agilitynetworks.com/what-is-the-cloud/cloud-computing-versus-traditional-server-installation-pro-and-con.html}
\bibitem{ref2}
	Rackspace,
	\emph{Understanding the Cloud Computing Stack: SaaS, PaaS, IaaS,}
	\url{http://www.rackspace.com/knowledge_center/whitepaper/understanding-the-cloud-computing-stack-saas-paas-iaas}
\bibitem{ref3}
	The Open Group,
	\emph{Cloud Computing Portability and Interoperability: Cloud
		Portability and Interoperability,}
	\url{http://www.opengroup.org/cloud/cloud/cloud_iop/cloud_port.htm}
\bibitem{ref4}
	Microsoft Cyber Trust Blog,
	\emph{Fundamentals of Cloud Service Reliability,}
	\url{http://blogs.microsoft.com/cybertrust/2012/09/12/fundamentals-of-cloud-service-reliability}
\bibitem{ref5}
	Wikipedia,
	\emph{Cloud computing security,}
	\url{http://en.wikipedia.org/wiki/Cloud_computing_security}
\bibitem{ref6}
	The Economic Times,
	\emph{Amazon links 8,000 Indian firms to Cloud services,}
	\url{http://articles.economictimes.indiatimes.com/2014-08-21/news/53072779_1_public-cloud-cloud-market-amazon-cloud}
\bibitem{ref7}
	Sudhir Murthy, Manager -- Cloud Services, Wipro India,
	\emph{Business Impact of Cloud Computing in India,}
	\url{http://tejas.iimb.ac.in/interviews/39.php}
\end{thebibliography}
\end{document}
